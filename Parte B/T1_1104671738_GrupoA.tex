\documentclass[9pt]{article}
\usepackage{makeidx}
\usepackage{multirow}
\usepackage{multicol}
\usepackage[dvipsnames,svgnames,table]{xcolor}
\usepackage{graphicx}
\usepackage{epstopdf}
\usepackage{ulem}
\usepackage{hyperref}
\usepackage{amsmath}
\usepackage{amssymb}
\usepackage[utf8]{inputenc} 

\author{Ricardo Jumbo}
\title{}
\usepackage[paperwidth=595pt,paperheight=842pt,top=70pt,right=85pt,bottom=70pt,left=85pt]{geometry}

\makeatletter
	\newenvironment{indentation}[3]%
	{\par\setlength{\parindent}{#3}
	\setlength{\leftmargin}{#1}       \setlength{\rightmargin}{#2}%
	\advance\linewidth -\leftmargin       \advance\linewidth -\rightmargin%
	\advance\@totalleftmargin\leftmargin  \@setpar{{\@@par}}%
	\parshape 1\@totalleftmargin \linewidth\ignorespaces}{\par}%
\makeatother 

% new LaTeX commands


\usepackage{Sweave}
\begin{document}
\Sconcordance{concordance:T1_1104671738_GrupoA.tex:T1_1104671738_GrupoA.Rnw:%
1 29 1 1 0 88 1 1 4 33 0 1 2 2 1 1 3 1 0 1 1 7 0 1 2 1 1 1 3 1 0 1 1 3 %
0 1 2 9 1}



	\begin{center}
		\includegraphics[width=80pt,height=80pt]{unl.png}
	\end{center}
	
	
	\begin{center}
		% W2L: warn: inserting end tag WordToLatex.WLFontStyle
		% W2L: warn: inserting end tag WordToLatex.WLFontStyle
		\begin{center}
			{\LARGE \textbf{\\
					UNIVERSIDAD NACIONAL DE LOJA}}
		\end{center}
		
	\end{center}
	
	\begin{center}
		\textbf{\\
			{
				\large \'AREA DE LA ENERG\'IA LAS INDUSTRIAS Y LOS
				RECURSOS NATURALES NO RENOVABLES}\\
		}
	\end{center}
	
	\begin{center}
		\textbf{\\
			{\Large CARRERA DE INGENIER\'IA EN SISTEMAS}}
	\end{center}
	
	\begin{center}
		\textbf{\\
			{\Large Examen }}
	\end{center}
	
	\textbf{\\
		{\Large TEMA:}}
	
	\begin{center}
		{\Large An\'alisis de Datos con Latex y R }
	\end{center}
	
	\textbf{\\
		{\Large Nombre: }}
	
	\begin{itemize}	
		\item {\Large Ricardo Jumbo}\\
		
		{\Large AFINF7369  }
	\end{itemize}
	
	\textbf{\\
		{\Large Paralelo:}\\
	}
	
	{\Large 6 ``A''}
	
	\textbf{\\
		{\Large Docente:}\\
	}
	
	{\Large Ing. Pablo Ordo\~nez}\\
	
	
	\begin{center}
		
		{\Large * 09 de diciembre del 2015*}
	\end{center}
	
\newpage

\begin{center}
\textbf{{\huge Respuestas}}

\end{center}

\begin{itemize}
		\item ?`Es aplicable la ingenieria de sofware cuando se elaboran webapps?`Si es asi, ?`c<f3>mo puede que asimile las caracteristicas unicas de estas\\\\
	efecitivamente se peude apliacar la ingenieria de software en la webapp ya que las caracteristicas unicas que esatas nos brindan nos permite hacer un uzo correcto de los estandares de analisis, contruccion entre otros que nos permite entregar un software ejecutable con alta calidad. 
	
	
     \item Una breve descripcion del dataset Titanic\\\\
     Este conjunto de datos proporciona informacion sobre el destino de los pasajeros en el primer viaje fatal del trasatlantico Titanic, que se resumen de acuerdo a la situacion economica (clase), el sexo, la edad y la supervivencia.
     
   
     
	\item Mostrar el dataset\\
, , Age = Child, Survived = No

      Sex
Class  Male Female
  1st     0      0
  2nd     0      0
  3rd    35     17
  Crew    0      0

, , Age = Adult, Survived = No

      Sex
Class  Male Female
  1st   118      4
  2nd   154     13
  3rd   387     89
  Crew  670      3

, , Age = Child, Survived = Yes

      Sex
Class  Male Female
  1st     5      1
  2nd    11     13
  3rd    13     14
  Crew    0      0

, , Age = Adult, Survived = Yes

      Sex
Class  Male Female
  1st    57    140
  2nd    14     80
  3rd    75     76
  Crew  192     20	
	\item ?`Cual es el numero total de casos en el dataset\\
	
\begin{Schunk}
\begin{Sinput}
> require(utils)
> summary(Titanic)
\end{Sinput}
Number of cases in table: 2201 
Number of factors: 4 
Test for independence of all factors:
	Chisq = 1637.4, df = 25, p-value = 0
	Chi-squared approximation may be incorrect\end{Schunk}
\\
 \textbf{El numero de casos es 2201}
\begin{Schunk}
\begin{Sinput}
> require(utils)
> sum(Titanic)
\end{Sinput}
[1] 2201\end{Schunk}

\end{itemize}


\newpage
\begin{center}
	\includegraphics [width=80pt,height=50pt]{creative}\\
\end{center}	

\end{document}
